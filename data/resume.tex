% !TeX root = ../sustechthesis-example.tex

\begin{resume}

  \section*{个人简历} % 根据正文撰写语言选择


  2018年9月——2022年6月,在武汉大学计算机学院学习,获得学士学位。

  2023年9月——2026年6月,在南方科技大学学习并攻读计算机科学硕士学位。

  获奖情况:国家奖学金、南科大优秀研究生、中国服务计算一等奖、先进院优秀学生等。


  % \section*{Resume} % 根据正文撰写语言选择
  % FamilyName GivenName was born in 1997, in Shenzhen, Guangdong, China.

  % In September 2015, he/she was admitted to Southern University of Science and Technology (SUSTech). In June 2019, he/she obtained a bachelor's degree in engineering from the Department of Computer Science and Engineering, SUSTech.【注:此行填写已获得的本科学士学位】

  % In September 2019, he/she began his/her graduate study in the Department of Computer Science and Engineering, SUSTech, and got a master of engineering degree in Electronic Science and Technology, in July 2022.【注:未获得硕士学位的学生无需此行】

  % Since September 2022, he/she has started to pursue his/her master/doctor's degree of engineering in Electronic Science and Technology in the Department of Computer Science and Engineering, SUSTech.【注:此行填写正在攻读学位】

  % Awards: XXXX scholarship, SUSTech, 2019.

  % Work experience: XXXX Corp., Software engineer Intern (June 2021 - August 2021); XXXX Corp., Software engineer Intern (June 2021 - August 2021).

  \section*{在学期间完成的相关学术成果}
  % \section*{Academic Achievements during the Study for an Academic Degree}

  \subsection*{学术论文}

  \begin{achievements}
    \item \textbf{He Y}, Xu M, Wu J, et al. UELLM: A Unified and Efficient Approach for
    Large Language Model Inference Serving[C]//Proceedings of the 22nd
    International Conference on Service-Oriented Computing (ICSOC).
    Springer, 2024: 218--235.
    (CCF-B,EI收录,对应学位论文第三章)

    \item \textbf{He Y}, Xu M, Wu J, et al. BanaServe: Unified KV Cache and Dynamic
    Module Migration for Balancing Disaggregated LLM Serving in AI
    Infrastructure[J]. Software: Practice and Experience, 2025.
    DOI: 10.1002/spe.70054.
    (CCF-B,SCI收录,对应学位论文第四章)

    \item Dang Y, \textbf{He Y}, Xu M, et al. Resource Management for GPT-based
    Model Deployed on Clouds: Challenges, Solutions, and Future
    Directions[C]//Proceedings of the 24th International Conference on
    Algorithms and Architectures for Parallel Processing (ICA3PP).
    2024.
    (CCF-C,EI收录)

    \item Wu J, Xu M, \textbf{He Y}, et al. CloudNativeSim: a Toolkit for Modeling
    and Simulation of Cloud-Native Applications[J]. Software: Practice
    and Experience, 2025.
    (CCF-B,SCI收录)

  \end{achievements}



  
  \subsection*{参与的科研项目及获奖情况}
  \begin{achievements}
    \item 2024年国家奖学金。
    \item CCF 2025 中国服务计算创新大赛一等奖
    \item 2024年南方科技大学优秀研究生。
    \item 2025年南方科技大学优秀研究生。
    \item 2024年深圳先进技术研究院数字所“优秀学生奖”。
    \item 2025年深圳先进技术研究院数字所“优秀学生奖”。
  \end{achievements}

\end{resume}
