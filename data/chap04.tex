% !TeX root = ../sustechthesis-example.tex

\chapter{面向PD分离架构的动态资源协同优化方法:BanaServe}

\section{引言}

随着大语言模型在生产环境中的大规模部署,推理服务系统正面临前所未有的资源管理挑战。以Prefill-Decode(PD)分离为代表的新型架构通过将提示预填充(Prefill)阶段与自回归解码(Decode)阶段分配至不同GPU实例,有效消除了两阶段间的资源竞争干扰。然而,现有PD分离系统在实际动态工作负载下暴露出三类根本性局限:

\begin{enumerate}
    \item \textbf{静态资源配置无法适应非平稳工作负载}:当请求速率较低(RPS $\leq$ 10)时,Prefill和Decode实例均存在20\%--40\%的GPU资源空闲浪费;而在突发高流量时,固定配置又成为吞吐量瓶颈;
    \item \textbf{PD两阶段间存在内生的资源利用率不均衡}:Prefill阶段计算密集(计算利用率约95\%,显存利用率约35\%),而Decode阶段恰好相反(计算利用率约35\%,显存利用率约90\%),这种互补性资源失衡在现有系统中无法被动态利用;
    \item \textbf{前缀缓存感知路由引发持续性负载倾斜}:以SGLang\cite{zheng2024sglang}和vLLM\cite{kwon2023efficient}为代表的系统在路由请求时优先考虑KV Cache命中率,导致命中率高的Prefill实例持续吸引更多请求,形成计算热点,而其他实例资源闲置。
\end{enumerate}

上述三类局限共享同一根本原因:\textbf{资源分配与状态(缓存)放置的紧耦合}。现有系统中,计算资源与KV Cache等状态数据绑定于特定实例,难以在不引入巨大开销的前提下动态迁移,导致负载均衡与缓存局部性之间存在不可调和的矛盾。

针对上述挑战,本章提出\textbf{BanaServe},一个面向PD分离架构的动态编排框架,通过三项核心创新将资源分配与状态管理解耦:

\begin{itemize}
    \item \textbf{层级权重迁移}(Layer-level Weight Migration):在Transformer层粒度上动态迁移模型权重,实现Prefill与Decode实例之间粗粒度的计算和显存资源再平衡;
    \item \textbf{注意力级KV Cache迁移}(Attention-level KV Cache Migration):沿注意力头维度分割KV Cache,将部分KV状态迁移至空闲GPU进行并行计算,实现细粒度的负载均衡;
    \item \textbf{全局KV Cache存储}(Global KV Cache Store):构建跨所有Prefill实例的统一KV Cache存储层,结合层级流水线重叠传输技术,使路由器摆脱缓存位置约束,实现纯负载感知调度。
\end{itemize}

BanaServe在vLLM和DistServe等主流推理栈上实现,并在Alpaca短文本和LongBench长文本等多种负载场景下进行评估。相比vLLM,BanaServe吞吐量提升1.2倍至3.9倍,总处理时间降低3.9\%至78.4\%;相比DistServe,吞吐量提升1.1倍至2.8倍,延迟降低1.4\%至70.1\%\cite{he2025banaserve}。

本章其余部分组织如下:第~\ref{sec:bana_motivation}节分析设计动机;第~\ref{sec:bana_design}节阐述系统设计;第~\ref{sec:bana_model}节建立性能优化模型;第~\ref{sec:bana_algo}节详述核心算法;第~\ref{sec:bana_eval}节给出实验评估;第~\ref{sec:bana_summary}节总结本章工作。

\section{问题分析与设计动机}
\label{sec:bana_motivation}

\subsection{静态配置导致的资源低效利用}

现有LLM推理系统在静态分配策略下频繁出现资源利用率不足的问题,其根源在于两个方面:其一,在低请求率(RPS $\leq$ 10)条件下,系统大量计算和显存资源处于闲置状态;其二,固定资源分配与LLM推理动态工作负载特性之间存在根本性的失配——满足存储需求往往制约可用的计算能力,反之亦然。

图~\ref{fig:bana_gpu_util}展示了在单台A100 GPU上部署LLaMA-13B模型时,Hugging Face Transformers(HFT)和vLLM在不同请求速率下的GPU资源利用率对比。实测结果表明,在低负载(RPS $\leq$ 10)条件下,两个系统均存在约20\%--40\%的GPU资源空闲浪费,验证了静态配置在资源效率方面的固有缺陷。

\begin{figure}[htbp]
    \centering
    \includegraphics[width=0.8\textwidth]{figures/bg1.pdf}
    \caption{HFT和vLLM在不同请求速率下的GPU资源利用率对比(基于单台A100 GPU部署LLaMA-13B模型,每组实验重复5次)。在低负载(RPS $\leq$ 10)场景下,两个系统均存在约20\%--40\%的GPU资源空闲浪费。}
    \label{fig:bana_gpu_util}
\end{figure}

\subsection{前缀缓存感知路由引发的负载倾斜}

图~\ref{fig:bana_cache_imbalance}展示了前缀缓存感知路由器在三个推理实例间分配请求的典型场景。该策略同时考虑缓存命中率和当前负载,但系统性地偏向命中率高的实例,导致持续性的负载倾斜:

\begin{itemize}
    \item \textbf{实例1}:命中率最高,接收大多数查询,计算利用率达100\%,缓存前缀Q1--Q5持续扩大,形成正反馈;
    \item \textbf{实例2}:接收较少查询,计算利用率仅40\%,存储的前缀Q6--Q7与其他实例存在冗余;
    \item \textbf{实例3}:计算利用率60\%,主要处理未缓存前缀Q8--Q10,需重复计算,丧失缓存收益。
\end{itemize}

这种分配模式形成正反馈动态:命中率高的实例吸引更多请求并扩大缓存内容,而低命中率实例处理较少查询却存储冗余数据或执行重复计算,最终导致排队延迟增加,系统整体吞吐量下降。

\begin{figure}[htbp]
    \centering
    \subcaptionbox{前缀缓存感知路由导致的负载倾斜\label{fig:bana_cache_skew}}
        {\includegraphics[width=0.48\textwidth]{figures/prefix_cache_router.png}}
    \hfill
    \subcaptionbox{PD分离架构中的资源利用率不均衡\label{fig:bana_pd_imbalance}}
        {\includegraphics[width=0.48\textwidth]{figures/pd_imbalance.png}}
    \caption{现有LLM推理架构的两类实测局限(。(a) 前缀缓存感知路由导致命中率高的实例持续过载,形成负载倾斜、冗余存储和重复计算;(b) PD分离架构下Prefill实例计算密集(约95\%)但显存闲置(约35\%),Decode实例恰好相反,资源利用率高度互补但不均衡。}
    \label{fig:bana_cache_imbalance}
\end{figure}

\subsection{PD分离架构的内生资源不均衡}

如图~\ref{fig:bana_pd_imbalance}所示,基于DistServe框架、LLaMA-13B模型和Alpaca数据集的实测结果表明:

\begin{itemize}
    \item \textbf{Prefill实例}:处理长提示序列,计算密集,通常维持超过95\%的计算利用率,但显存利用率仅约35\%;
    \item \textbf{Decode实例}:执行迭代式自回归生成,利用Prefill阶段生成的KV Cache降低计算量,计算利用率约35\%,但显存利用率显著更高;
    \item \textbf{通信特征}:Prefill实例需向Decode实例传输KV Cache数据,导致单向带宽占主导,其他通信方向利用率低下。
\end{itemize}

这种计算-显存资源利用率的系统性不均衡在现有PD分离系统中无法通过动态机制加以利用,成为制约整体资源效率的核心瓶颈。

\section{系统设计}
\label{sec:bana_design}

BanaServe的整体架构由三个核心模块协同构成:

\begin{itemize}
    \item \textbf{负载感知请求调度器}:接收传入请求队列,基于各Prefill实例的实时归一化综合利用率(计算+显存)进行排序,将请求分配至负载最低的实例,摆脱对缓存位置的依赖;
    \item \textbf{全局KV Cache存储}:由CPU内存和SSD共同构成存储层次,支持跨所有Prefill实例的统一KV Cache共享,结合三阶段层级流水线重叠传输技术(预取-计算-存储并行),将通信延迟隐藏于计算过程中;
    \item \textbf{动态模块迁移引擎}:周期性监控各GPU实例的实时负载,根据负载差距$\Delta$和迁移效率比$\rho$自适应选择层级权重迁移(粗粒度)或注意力级KV Cache迁移(细粒度),实现Prefill与Decode实例之间的动态资源再平衡。
\end{itemize}

三个模块协同工作:调度器的纯负载感知决策由全局KV Cache存储提供支撑(消除缓存局部性约束),迁移引擎则持续监控并调整系统资源分配,形成闭环的自适应优化机制。

\subsection{层级权重迁移}

\textbf{1. 设计原理}

在工作负载不均衡程度较高的场景下,BanaServe将一组连续的Transformer层从一个GPU实例迁移至另一个GPU实例。这种粗粒度的重分配实现了动态模型并行,通过同时迁移层权重和对应的KV Cache,在Prefill与Decode阶段之间重新分配大量计算和显存资源。

如图~\ref{fig:bana_layer_migration}所示,设迁移层的权重大小为$S^w_\ell$,对应KV Cache大小为$S^{kv}_\ell$,总迁移数据量为:
\begin{equation}
    S^{\text{total}}_\ell = S^w_\ell + S^{kv}_\ell
    \label{eq:layer_migration_size}
\end{equation}
给定有效互联带宽$B_{\text{net}}$和同步开销$T_{\text{sync}}$,层级迁移延迟近似为:
\begin{equation}
    T_{\text{layer}} \approx \frac{S^{\text{total}}_\ell}{B_{\text{net}}} + T_{\text{sync}}
    \label{eq:layer_migration_latency}
\end{equation}
由于$S^w_\ell \gg S^{kv}_\ell$,延迟主要由权重传输主导,但在高带宽数据中心互联(NVLink、InfiniBand)环境下仍保持实用性。

\textbf{2. 执行正确性保证}

对于已迁移的层$f_\ell(\cdot)$,其权重$W_\ell$和KV Cache $\mathcal{K}_\ell$同步迁移至目标GPU后,计算语义保持不变:
\begin{equation}
    \mathbf{y}_\ell = f_\ell(\mathbf{x}_\ell; W_\ell, \mathcal{K}_\ell)
    \label{eq:layer_correctness}
\end{equation}
其中$\mathbf{x}_\ell$为上一层的输入激活。通过同步传输$(W_\ell, \mathcal{K}_\ell)$,确保迁移后自回归解码可以无缝继续,不产生计算结果的语义变化。

\begin{figure}[htbp]
    \centering
    \subcaptionbox{层级权重迁移\label{fig:bana_layer_migration}}
        {\includegraphics[width=0.48\textwidth]{figures/offload.pdf}}
    \hfill
    \subcaptionbox{注意力级KV Cache迁移\label{fig:bana_attn_migration}}
        {\includegraphics[width=0.48\textwidth]{figures/attention_offload.png}}
    \caption{BanaServe的两种迁移粒度。(a) 层级迁移将连续Transformer层(含权重$W_\ell$和KV Cache $\mathcal{K}_\ell$)从高负载GPU迁移至低负载GPU,实现粗粒度负载均衡;(b) 注意力级迁移沿注意力头维度分割KV Cache,将$K^{(2)},V^{(2)}$卸载至冷GPU并行计算,仅交换极少量中间结果,实现细粒度负载均衡。}
    \label{fig:bana_migration}
\end{figure}

\subsection{注意力级KV Cache迁移}

\textbf{1. 设计原理}

当需要细粒度负载调整时,BanaServe沿注意力头维度对KV Cache进行分割,将部分头的KV状态选择性地迁移至空闲的冷GPU(Cold GPU)。由于不传输任何模型权重,仅移动中间的键值激活,该方法引入的迁移开销极小。

如图~\ref{fig:bana_attn_migration}所示,流程从QKV投影层开始,生成查询矩阵$Q$、键矩阵$K$和值矩阵$V$,随后将注意力头划分为两个不相交子集:
\begin{itemize}
    \item $K^{(1)},V^{(1)}$(KV1):保留在热GPU(Hot GPU)上进行本地处理;
    \item $K^{(2)},V^{(2)}$(KV2):卸载至冷GPU进行远程并行处理。
\end{itemize}

\textbf{2. 并行计算可行性分析}

设$H$为注意力头总数,$d$为头维度,$Q \in \mathbb{R}^{B_q \times Hd}$为查询矩阵,$K^{(j)}, V^{(j)} \in \mathbb{R}^{B_k \times d h_j}$为分配给设备$j \in \{1,2\}$的键值子集。各子集的注意力分数和值聚合分别计算为:
\begin{equation}
    S^{(j)} = Q \cdot (K^{(j)})^\top
    \label{eq:attn_score}
\end{equation}
\begin{equation}
    A^{(j)} = \exp\!\left(S^{(j)}\right)
    \label{eq:attn_softmax}
\end{equation}
\begin{equation}
    \ell^{(j)} = \begin{cases}
        \sum_i A^{(1)}_i, & j = 1 \\
        \ell^{(1)} + \sum_i A^{(2)}_i, & j = 2
    \end{cases}
    \label{eq:attn_norm}
\end{equation}
\begin{equation}
    O^{(j)} = \frac{A^{(j)}}{\ell^{(j)}} \cdot V^{(j)}
    \label{eq:attn_output}
\end{equation}
最终注意力输出为:
\begin{equation}
    O = O^{(1)} + O^{(2)}
    \label{eq:attn_final}
\end{equation}
由于$K^{(1)},V^{(1)}$和$K^{(2)},V^{(2)}$来自不相交的头分区,$S^{(1)}$和$S^{(2)}$可在两个设备上完全并行计算,无中间数据依赖。跨设备仅需传输$\ell^{(1)}$(归一化因子)和$O^{(1)}$,确保全局softmax的正确性,同时将传输量控制在极小范围内。

\textbf{3. 迁移延迟分析}

设迁移的KV数据大小为$S_{kv}$,网络带宽为$B_{\text{net}}$,则注意力级迁移延迟为:
\begin{equation}
    T_{\text{attn}} \approx \frac{S_{kv}}{B_{\text{net}}}
    \label{eq:attn_migration_latency}
\end{equation}
由于$S_{kv} \ll S_\ell$(层级状态大小),因此$T_{\text{attn}} \ll T_{\text{layer}}$,注意力级迁移具有轻量、低延迟的特性,特别适合异构解码工作负载的实时负载均衡。

通过将粗粒度层迁移与细粒度注意力迁移相结合,BanaServe可在运行时自适应选择最优策略,在迁移开销($T_{\text{mig}}$)与利用率收益(最小化$\max_g U_g$)之间权衡,同时满足延迟预算约束$T_{\text{mig}} \leq T_{\text{budget}}$。

\subsection{全局KV Cache存储}

\textbf{1. 设计原理}

前缀缓存感知路由器的核心瓶颈在于前缀缓存存储在Prefill节点间的不均衡分布,导致路由器必须同时考虑计算负载和缓存位置。为解决这一问题,BanaServe引入\textbf{全局KV Cache存储}(Global KV Cache Store),跨所有Prefill和Decode节点共享KV Cache。

如图~\ref{fig:bana_kvcache_store}所示,全局KV Cache存储层由CPU内存和SSD共同构成存储层次,所有Prefill节点均可访问统一的KV Cache存储。在此设计下,路由器只需关注计算负载均衡,所有前缀缓存的加载和存取操作均委托给KV Cache Store处理,从根本上消除了缓存局部性对路由决策的约束。

\begin{figure}[htbp]
    \centering
    \includegraphics[width=\textwidth]{figures/kv_store.png}
    \caption{BanaServe全局KV Cache存储的管理与复用机制。Prefill实例对传入请求执行增量预填充,复用已缓存的前缀KV Cache;新生成的KV Cache存储于共享的CPU/SSD支持的KV存储中;Decode实例从共享存储中检索完整KV Cache用于token生成,避免重复计算。}
    \label{fig:bana_kvcache_store}
\end{figure}

\textbf{2. 层级流水线重叠传输}

全局KV Cache存储面临的主要挑战是频繁加载和存取操作引入的延迟。BanaServe利用Transformer模型逐层执行的特性,设计了三阶段流水线,将预取、计算和加载操作相互重叠,隐藏大部分内存访问延迟。

设$T_F$为Prefill阶段的总前向计算时间,$r$为平均前缀缓存命中率,$L$为输入序列长度(token),$N$为Transformer层数。每层前向计算时间为:
\begin{equation}
    T_{F,\text{layer}} = \frac{T_F \cdot r}{N}
    \label{eq:layer_compute_time}
\end{equation}
设$S_{kv}$为单token的每层KV Cache大小(字节),$B$为有效PCIe带宽,每层KV Cache传输时间为:
\begin{equation}
    T_{KV} = \frac{S_{kv} \cdot L \cdot r}{B}
    \label{eq:kv_transfer_time}
\end{equation}
以LLaMA-3.1-8B模型为例(隐藏维度$d_{\text{model}} = 4096$,总注意力头$h_{\text{total}} = 32$,GQA下KV头$h_{kv} = 8$,BF16精度),每头维度为:
\begin{equation}
    d_{\text{head}} = \frac{d_{\text{model}}}{h_{\text{total}}} = 128
    \label{eq:head_dim}
\end{equation}
每层KV Cache大小(每个KV头存储键和值两组激活)为:
\begin{equation}
    S_{kv} = h_{kv} \cdot d_{\text{head}} \cdot 2 \cdot 2\ \text{字节} = 8 \times 128 \times 2 \times 2 = 4096\ \text{字节(4 KB)}
    \label{eq:kv_size}
\end{equation}
乘以$N = 32$层,每token的总KV Cache大小为$S_{kv,\text{total}} = 32 \times 4\ \text{KB} = 128\ \text{KB}$。

在$L = 1000$ tokens、$r = 0.5$、$B = 200$ Gbps、$T_F = 270$ ms的实验条件下:
\begin{equation}
    T_{F,\text{layer}} = \frac{270\ \text{ms} \times 0.5}{32} \approx 4.22\ \text{ms}, \quad
    T_{KV} = \frac{4\ \text{KB} \times 1000 \times 0.5}{200\ \text{Gbps}} \approx 0.082\ \text{ms}
    \label{eq:overlap_validation}
\end{equation}
由于$T_{KV} \ll T_{F,\text{layer}}$,KV Cache传输延迟完全可被前向计算所覆盖,实现通信与计算的有效重叠。图~\ref{fig:bana_pipeline}展示了三阶段流水线的执行时序:在GPU执行第$L_i$层前向计算的同时,HtoD通道预取第$L_{i+1}$层的KV Cache,DtoH通道存储第$L_{i-1}$层的缓存,三个阶段并行执行,有效隐藏通信延迟。

\begin{figure}[htbp]
    \centering
    \includegraphics[width=0.9\textwidth]{figures/kv_transfer.png}
    \caption{BanaServe三阶段层级KV Cache流水线验证。在50\%平均前缀缓存命中率下,每层前向计算时间(4.22 ms)远大于每层KV Cache传输时间(0.082 ms),确保有效重叠。时序图展示三层($L_1$--$L_3$)的并发执行:GPU执行$L_i$的前向计算时,HtoD通道预取$L_{i+1}$的KV Cache,DtoH通道存储$L_{i-1}$的缓存。}
    \label{fig:bana_pipeline}
\end{figure}

\section{性能优化模型}
\label{sec:bana_model}

\subsection{优化目标}

BanaServe将PD分离系统的性能优化建模为多目标问题,目标是最大化资源利用率、最小化端到端延迟并最大化吞吐量。

设$\mathcal{P}$和$\mathcal{D}$分别为Prefill和Decode GPU集合,$U_g$为GPU $g$的利用率,$T_{\text{TTFT}}$为首token延迟,$T_{\text{TPOT}}$为每输出token时间,$\Theta$为系统有效吞吐量(tokens/s)。联合优化目标定义为:
\begin{equation}
    \max_\pi\ \alpha \cdot U_{\text{avg}}(\pi) - \beta \cdot T_{\text{avg-latency}}(\pi) + \gamma \cdot \Theta(\pi)
    \label{eq:joint_objective}
\end{equation}
其中:
\begin{equation}
    U_{\text{avg}}(\pi) = \frac{1}{|\mathcal{P} \cup \mathcal{D}|} \sum_g U_g(\pi), \quad
    T_{\text{avg-latency}} = \frac{1}{N} \sum_{i=1}^{N} \left(T^{(i)}_{\text{TTFT}} + T^{(i)}_{\text{TPOT}}\right)
    \label{eq:utilization_latency}
\end{equation}
$\alpha, \beta, \gamma$为利用率、延迟和吞吐量三项的权重系数,$\pi$为当前模块迁移方案。

\subsection{延迟模型}

首token延迟(TTFT)建模为:
\begin{equation}
    T_{\text{TTFT}} = T_p + T_x + T_q
    \label{eq:ttft_model}
\end{equation}
其中$T_p$为Prefill计算时间,$T_x = T_{\text{load}} + T_{\text{fetch}}$为KV Cache传输时间(包含从存储加载时间和从远程GPU获取时间),$T_q$为Decode前的排队延迟。

每输出token时间(TPOT)建模为:
\begin{equation}
    T_{\text{TPOT}} = T_d + T_c + T_m
    \label{eq:tpot_model}
\end{equation}
其中$T_d$为基础解码计算时间,$T_c$为缓存访问延迟,$T_m$为内存带宽停顿时间。

\subsection{资源利用率模型}

对于具有$n_p$个Transformer层的Prefill实例,设$M_0$为进程基础内存开销,$M_\ell$为每层内存,$K_{\text{init}}$为初始KV Cache分配,总显存占用为:
\begin{equation}
    \text{Mem}_p = M_0 + n_p \cdot M_\ell + K_{\text{init}}
    \label{eq:prefill_mem}
\end{equation}
设$C_\ell$为每层每token的计算开销,$B_{sz}$为批大小,$L_{\text{in}}$为输入长度,总计算需求为:
\begin{equation}
    \text{Comp}_p = n_p \cdot C_\ell \cdot B_{sz} \cdot L_{\text{in}}
    \label{eq:prefill_comp}
\end{equation}
类似地,对于具有$n_d$个Transformer层的Decode实例,设$K_{\text{acc}}$为解码过程中累积的KV Cache大小,$L_{\text{gen}}$为生成长度:
\begin{equation}
    \text{Mem}_d = M_0 + n_d \cdot M_\ell + K_{\text{acc}}
    \label{eq:decode_mem}
\end{equation}
\begin{equation}
    \text{Comp}_d = n_d \cdot C_\ell \cdot B_{sz} \cdot L_{\text{gen}}
    \label{eq:decode_comp}
\end{equation}
给定GPU峰值计算能力$C_{\text{gpu}}$,各阶段的平均利用率为:
\begin{equation}
    U_p = \frac{\text{Comp}_p}{C_{\text{gpu}}}, \quad U_d = \frac{\text{Comp}_d}{C_{\text{gpu}}}
    \label{eq:stage_utilization}
\end{equation}

\subsection{迁移开销模型}

迁移$k$个模块所引入的总开销为:
\begin{equation}
    \text{Cost}_{\text{mig}} = k \cdot \left(T_{x\_\text{lat}} + T_{\text{sync}} + T_{\text{mem\_realloc}}\right)
    \label{eq:migration_cost}
\end{equation}
其中$T_{x\_\text{lat}}$为KV Cache加权重的传输延迟,$T_{\text{sync}}$为与正在进行的计算同步所需时间,$T_{\text{mem\_realloc}}$为缓冲区重分配时间。

模块迁移方案$\pi$须满足延迟约束:
\begin{equation}
    T_{\text{TTFT}}(\pi) \leq T^{\text{budget}}_{\text{TTFT}}, \quad
    T_{\text{TPOT}}(\pi) \leq T^{\text{budget}}_{\text{TPOT}}
    \label{eq:latency_constraint}
\end{equation}

\subsection{系统吞吐量}

给定$N$个并发请求和每请求总输出长度$L_{\text{out}}$(token),系统吞吐量为:
\begin{equation}
    \Theta = \frac{N \cdot L_{\text{out}}}{T_{\text{TTFT}} + L_{\text{out}} \cdot T_{\text{TPOT}}}
    \label{eq:throughput}
\end{equation}
调度器周期性地求解最优迁移方案:
\begin{equation}
    \pi^* = \arg\max_\pi \left(\alpha U_{\text{avg}}(\pi) - \beta T_{\text{avg\_latency}}(\pi) + \gamma \Theta(\pi)\right)
    \label{eq:optimal_plan}
\end{equation}
在满足迁移开销和延迟约束的前提下执行。

\section{核心算法设计}
\label{sec:bana_algo}

基于上述性能模型,BanaServe实现两个互补的运行时算法:自适应模块迁移算法(Dynamic Migration Algorithm)和负载感知请求调度算法(Load-aware Request Scheduling Algorithm)。

\subsection{自适应模块迁移算法}

\textbf{1. 算法设计}

自适应模块迁移算法通过周期性监控实时负载并执行模块迁移来均衡工作负载,同时限制迁移开销。设$D = \{d_1, d_2, \ldots, d_{|D|}\}$为当前参与服务的GPU集合,$\delta > 0$为负载不均衡阈值。对每个设备$d$,定义归一化利用率为:
\begin{equation}
    U_d = \frac{C_d}{C^{\max}_d} + \frac{M_d}{M^{\max}_d}
    \label{eq:normalized_util}
\end{equation}
其中$C_d$为当前计算使用量,$M_d$为当前显存使用量,$C^{\max}_d$和$M^{\max}_d$为对应的硬件容量上限。$U_d$的范围为$[0, 2.0]$,值越大表示综合计算和显存负载越重。

算法~\ref{algo:dynamic_migration}详述了完整的执行流程。

\begin{algorithm}[htbp]
    \caption{自适应模块迁移算法(BanaServe Dynamic Migration)}
    \label{algo:dynamic_migration}
    \KwIn{设备集合$D$;当前模块分配方案$cfg$;负载不均衡阈值$\delta$}
    \KwOut{更新后的模块分配方案$cfg'$}
    \BlankLine
    \tcp{阶段一:测量当前负载}
    \ForEach{$d \in D$}{
        $load[d] \leftarrow \text{MeasureUtilization}(d)$
            \tcp*{计算+显存综合利用率}
    }
    $overload \leftarrow \{d \in D \mid load[d] - \min(load) > \delta\}$\;
    $underload \leftarrow \{d \in D \mid \max(load) - load[d] > \delta\}$\;
    \BlankLine
    \tcp{阶段二:迁移决策循环}
    \While{$\exists\, d_o \in overload$ \textbf{且} $\exists\, d_u \in underload$}{
        $\Delta \leftarrow load[d_o] - load[d_u]$\;
        \uIf{$\Delta \geq \delta$ \textbf{且} $\text{SupportsLayerMigration}(d_o)$}{
            \tcp{粗粒度:层级权重迁移}
            $\text{MigrateLayer}(d_o, d_u)$\;
        }\uElseIf{$\Delta \geq \delta$ \textbf{且} $\text{SupportsAttentionMigration}(d_o)$}{
            \tcp{细粒度:注意力级KV Cache迁移}
            $\text{MigrateKVHeads}(d_o, d_u)$\;
        }
        更新$load[d_o]$和$load[d_u]$\;
        相应更新$cfg'$\;
    }
    \Return{$cfg'$}\;
\end{algorithm}

\textbf{2. 算法执行分四个阶段:}

\textbf{阶段一:负载测量(第1--3行)}

对每个设备$d \in D$,调用MeasureUtilization计算综合计算和显存利用率$load[d]$,获取各设备当前归一化负载,用于后续不均衡判断。

\textbf{阶段二:负载分类(第4--5行)}

通过以下条件构建过载和欠载设备集合:
\begin{equation}
    overload = \{d \in D \mid load[d] - \min(load) > \delta\}, \quad
    underload = \{d \in D \mid \max(load) - load[d] > \delta\}
    \label{eq:load_classification}
\end{equation}
过载设备的利用率超过最轻负载设备超过阈值$\delta$,欠载设备的利用率低于最重负载设备超过$\delta$。

\textbf{阶段三:迁移决策与执行(第6--13行)}

当存在过载设备$d_o$和欠载设备$d_u$时,计算瞬时负载差:
\begin{equation}
    \Delta = load[d_o] - load[d_u]
    \label{eq:load_gap}
\end{equation}
若$\Delta \geq \delta$且$d_o$支持层级迁移,调用MigrateLayer迁移连续Transformer层;若$\Delta \geq \delta$且$d_o$支持注意力级迁移,调用MigrateKVHeads迁移选定KV Cache段。执行前评估迁移的收益-开销比:
\begin{equation}
    \text{Benefit}(m) = \Delta_{\text{before}} - \Delta_{\text{after}}, \quad
    \text{Cost}(m) = T_{\text{transfer}}(m) + T_{\text{sync}}(m)
    \label{eq:benefit_cost}
\end{equation}
仅当$\text{Benefit}(m) / \text{Cost}(m) \geq \rho$(可配置效率比)时才执行迁移。

\textbf{阶段四:更新分配方案(第14--16行)}

每次迁移后更新$load[d_o]$和$load[d_u]$,修订$cfg'$以反映新的模块放置,重复直至不再存在同时满足过载和欠载条件的设备对,最终返回更新后的$cfg'$。

\textbf{3. 复杂度与收敛性分析}

每个控制周期的时间复杂度为:
\begin{equation}
    O(|D| + N_m)
    \label{eq:complexity}
\end{equation}
其中$|D|$来自负载测量和分类,$N_m$来自模块选择和开销评估。在典型部署中$|D|$为数十量级,$N_m$受模型深度约束,支持实时运行。为防止振荡,采用滞后控制,设置不同的上升阈值$\delta_\uparrow$和下降阈值$\delta_\downarrow$。全局KV Cache存储的集成确保迁移过程中所有KV状态得以保存,避免重新计算并维持稳定吞吐量。

\subsection{负载感知请求调度算法}

\textbf{1. 算法设计}

负载感知请求调度算法(算法~\ref{algo:load_aware_scheduling})将传入请求高效分配给Prefill实例。与缓存感知路由策略不同,由于全局KV Cache存储支持所有Prefill GPU之间的KV Cache共享,该调度器无需考虑前缀缓存命中率,调度决策完全基于实时负载指标。

对每个Prefill实例$p_i$,归一化负载定义为:
\begin{equation}
    U_{p_i} = \frac{C_{p_i}}{C^{\max}_{p_i}} + \frac{M_{p_i}}{M^{\max}_{p_i}}
    \label{eq:prefill_load}
\end{equation}
其中$C_{p_i}$和$M_{p_i}$为当前计算和显存使用量,$C^{\max}_{p_i}$和$M^{\max}_{p_i}$为对应容量上限。

\begin{algorithm}[htbp]
    \caption{负载感知请求调度算法(BanaServe Load-aware Scheduling)}
    \label{algo:load_aware_scheduling}
    \KwIn{请求队列$Q$;Prefill实例集合$P$;负载阈值$\delta_L$}
    \KwOut{调度方案$D^*$}
    \BlankLine
    \tcp{阶段一:测量每个Prefill实例的当前负载}
    \ForEach{$p_i \in P$}{
        $load[p_i] \leftarrow \text{MeasureUtilization}(p_i)$
            \tcp*{计算+显存}
        $q\_len[p_i] \leftarrow \text{GetQueueLength}(p_i)$\;
    }
    \BlankLine
    \tcp{阶段二:按负载和队列长度排序实例}
    $candidates \leftarrow \text{Sort}(P,\ key=(load, q\_len),\ order=\text{ascending})$\;
    \BlankLine
    \tcp{阶段三:将请求分配至负载最低的实例}
    \ForEach{$req \in Q$}{
        $p_{target} \leftarrow \text{Select}(candidates,\ policy=\text{least-loaded})$\;
        \uIf{$load[p_{target}] < \delta_L$}{
            $\text{AssignRequest}(req, p_{target})$\;
            $load[p_{target}] \leftarrow load[p_{target}] + \text{EstimateLoad}(req)$\;
        }\Else{
            $p_{target} \leftarrow \text{Select}(candidates,\ policy=\text{lowest-queue})$\;
            $\text{AssignRequest}(req, p_{target})$\;
        }
    }
    \BlankLine
    \tcp{阶段四:返回最终调度方案}
    $D^* \leftarrow \text{GetDispatchMap}(P)$\;
    \Return{$D^*$}\;
\end{algorithm}

\textbf{2. 算法执行分四个阶段:}

\textbf{阶段一:负载测量(第1--4行)}

对每个$p_i \in P$,调用MeasureUtilization测量$U_{p_i}$,并通过GetQueueLength记录队列长度$q\_len[p_i]$。

\textbf{阶段二:排序(第5行)}

按负载和队列长度升序排列所有实例,生成用于调度决策的候选列表。

\textbf{阶段三:调度循环(第6--13行)}

对每个请求$req \in Q$,从候选列表中选择负载最低的实例$p_{target}$。若$U_{p_{target}} < \delta_L$,将$req$分配给$p_{target}$并将该请求的估计负载贡献加至$load[p_{target}]$;否则,选择队列长度最小的实例并分配。

\textbf{阶段四:生成调度方案(第14--15行)}

调用GetDispatchMap生成调度映射$D^*$并返回至编排组件执行。

\textbf{3. 复杂度分析}

每个调度周期的时间复杂度为:
\begin{equation}
    O(|P| \log |P| + |Q|)
    \label{eq:scheduling_complexity}
\end{equation}
其中$O(|P| \log |P|)$来自按当前负载和队列长度对实例排序,$O(|Q|)$来自逐请求分配循环。在典型部署中$|P|$为数十量级(数十个GPU),算法适合实时调度。

通过全局KV Cache存储将缓存命中率从路由标准中剥离,该调度器专注于均衡负载和最小化排队延迟,有效降低了调度复杂度,避免Prefill节点热点形成,并在动态工作负载下确保稳定吞吐量。

\section{实验评估}
\label{sec:bana_eval}

\subsection{实验设置}

\textbf{评估模型。}选用两种具有代表性的13B参数级大语言模型,如表~\ref{tab:bana_models}所示。LLaMA-13B作为LLaMA系列的主要评估目标,用于评估BanaServe处理大规模纯解码器Transformer的能力;OPT-13B采用不同的架构优化和训练方法,用于跨架构的泛化性验证。

\begin{table}[htbp]
    \centering
    \caption{实验评估所用模型配置}
    \label{tab:bana_models}
    \begin{tabularx}{\textwidth}{p{3cm}AAA}
        \toprule
        \textbf{模型} & \textbf{参数量} & \textbf{架构} & \textbf{评估目的} \\
        \midrule
        LLaMA-13B & 13B & 纯解码器 & 系列内性能评估 \\
        OPT-13B   & 13B & 纯解码器 & 跨架构泛化验证 \\
        \bottomrule
    \end{tabularx}
\end{table}

\textbf{基准测试与工作负载。}实验设计采用两个互补的公开基准测试:

\begin{itemize}
    \item \textbf{Alpaca}\cite{alpaca}(短文本场景):包含52,000条指令跟随样例,输入序列长度集中在4--50 token,代表聊天机器人、代码助手和问答系统等交互式应用的典型生产推理工作负载;
    \item \textbf{LongBench}\cite{longbench}(长文本场景):覆盖21项多样化任务,上下文长度从约2,000 token跨越至超过85,000 token,涵盖多文档问答、文档摘要、少样本学习和代码补全等场景,对KV Cache管理和显存优化策略形成全面压力测试。
\end{itemize}

所有实验中最大输出长度上限设为512 token,以确保跨基准测试的一致性并聚焦于输入长度差异对系统性能的影响。

\textbf{负载测试方法。}请求速率从1到20 RPS(每秒请求数)系统性地递增,采用泊松到达过程模拟真实流量的突发特性。每组实验配置包含60秒预热期以确保系统稳定并填充缓存,所有实验重复5次(不同随机种子),报告均值及95\%置信区间。

\textbf{评估指标。}采用三项核心指标:吞吐量(requests/s)、总处理时间(s)和平均延迟(ms),分别衡量系统原始处理能力、端到端效率和用户感知响应性。

\textbf{对比基线。}与两个当前最优LLM推理系统进行对比:
\begin{itemize}
    \item \textbf{vLLM}\cite{kwon2023efficient}:学术界和工业界广泛采用的推理系统,基于连续批处理和PagedAttention实现高效KV Cache管理,但单体架构难以平衡Prefill和Decode阶段的竞争需求;
    \item \textbf{DistServe}\cite{zhong2024distserve}:将Prefill和Decode计算分配至专用实例的分离式推理系统,虽缓解了阶段间干扰,但引入了额外的通信开销和组件协调需求。
\end{itemize}

\subsection{短文本场景性能评估}

图~\ref{fig:bana_short_llama}和图~\ref{fig:bana_short_opt}分别展示了LLaMA-13B和OPT-13B在Alpaca短文本基准测试下的性能对比结果。

\begin{figure}[htbp]
    \centering
    \subcaptionbox{吞吐量\label{fig:bana_short_llama_thr}}
        {\includegraphics[width=0.32\textwidth]{figures/llama_short_throughput.pdf}}
    \hfill
    \subcaptionbox{总处理时间\label{fig:bana_short_llama_time}}
        {\includegraphics[width=0.32\textwidth]{figures/llama_short_total_time.pdf}}
    \hfill
    \subcaptionbox{平均延迟\label{fig:bana_short_llama_lat}}
        {\includegraphics[width=0.32\textwidth]{figures/llama_short_avg_latency.pdf}}
    \caption{LLaMA-13B短文本场景性能结果。在1--20 RPS的全请求速率范围内,BanaServe在吞吐量、总处理时间和平均延迟三项指标上均优于DistServe和vLLM,吞吐量提升1.1倍至1.2倍,总处理时间显著降低。}
    \label{fig:bana_short_llama}
\end{figure}

\begin{figure}[htbp]
    \centering
    \subcaptionbox{吞吐量\label{fig:bana_short_opt_thr}}
        {\includegraphics[width=0.32\textwidth]{figures/opt_short_throughput.pdf}}
    \hfill
    \subcaptionbox{总处理时间\label{fig:bana_short_opt_time}}
        {\includegraphics[width=0.32\textwidth]{figures/opt_short_total_time.pdf}}
    \hfill
    \subcaptionbox{平均延迟\label{fig:bana_short_opt_lat}}
        {\includegraphics[width=0.32\textwidth]{figures/opt_short_avg_latency.pdf}}
    \caption{OPT-13B短文本场景性能结果。BanaServe相比DistServe吞吐量提升2.8倍至3.9倍,相比vLLM最高提升3.9倍;平均延迟相比vLLM降低3.9\%至78.4\%,相比DistServe降低1.4\%至70.1\%。}
    \label{fig:bana_short_opt}
\end{figure}

\textbf{LLaMA-13B短文本分析:}在整个RPS范围内,BanaServe持续提供更高吞吐量(相比DistServe和vLLM提升1.1倍至1.2倍),同时保持更低延迟和更短总处理时间。这些改进源于BanaServe最小化的调度开销和高效的KV Cache处理机制,支持快速上下文切换而不引发显著的流水线停顿。

\textbf{OPT-13B短文本分析:}在OPT-13B上,性能提升更为显著。BanaServe吞吐量相比DistServe提升2.8倍至3.9倍,相比vLLM最高提升3.9倍;平均延迟相比vLLM降低3.9\%至78.4\%,相比DistServe降低1.4\%至70.1\%。这些结果凸显了BanaServe在高频上下文切换场景下的效率优势——其批处理和缓存复用策略使GPU利用率接近饱和同时保持响应性。

\subsection{长文本场景性能评估}

图~\ref{fig:bana_long_llama}和图~\ref{fig:bana_long_opt}分别展示了LLaMA-13B和OPT-13B在LongBench长文本基准测试下的性能对比结果。

\begin{figure}[htbp]
    \centering
    \subcaptionbox{吞吐量\label{fig:bana_long_llama_thr}}
        {\includegraphics[width=0.32\textwidth]{figures/llama_long_throughput.pdf}}
    \hfill
    \subcaptionbox{总处理时间\label{fig:bana_long_llama_time}}
        {\includegraphics[width=0.32\textwidth]{figures/llama_long_total_time.pdf}}
    \hfill
    \subcaptionbox{平均延迟\label{fig:bana_long_llama_lat}}
        {\includegraphics[width=0.32\textwidth]{figures/llama_long_avg_latency.pdf}}
    \caption{LLaMA-13B长文本场景性能结果。BanaServe在1--10 RPS范围内吞吐量提升1.3倍至1.5倍(相比DistServe和vLLM),延迟降低1.4\%至65.3\%(相比vLLM)。在高负载(10 RPS以上)场景下优势更为明显。}
    \label{fig:bana_long_llama}
\end{figure}

\begin{figure}[htbp]
    \centering
    \subcaptionbox{吞吐量\label{fig:bana_long_opt_thr}}
        {\includegraphics[width=0.32\textwidth]{figures/opt_long_throughput.pdf}}
    \hfill
    \subcaptionbox{总处理时间\label{fig:bana_long_opt_time}}
        {\includegraphics[width=0.32\textwidth]{figures/opt_long_total_time.pdf}}
    \hfill
    \subcaptionbox{平均延迟\label{fig:bana_long_opt_lat}}
        {\includegraphics[width=0.32\textwidth]{figures/opt_long_avg_latency.pdf}}
    \caption{OPT-13B长文本场景性能结果。BanaServe相比DistServe和vLLM吞吐量提升1.1倍至1.3倍,延迟改善模式与LLaMA-13B一致,在不同负载水平下均保持稳定的性能优势。}
    \label{fig:bana_long_opt}
\end{figure}

\textbf{LLaMA-13B长文本分析:}在LongBench扩展序列基准测试上,BanaServe相比DistServe和vLLM吞吐量提升1.3倍至1.5倍,延迟降低20.6\%至65.3\%(相比vLLM)和1.4\%至20.6\%(相比DistServe)。性能差距在高负载条件(10--20 RPS)下更为明显,此时传统系统受缓存争用和流水线阻塞影响,而BanaServe的统一架构减少了组件间通信开销,并通过动态批处理平衡各阶段的计算和显存资源。

\textbf{OPT-13B长文本分析:}在相同LongBench设置下,BanaServe相比DistServe和vLLM吞吐量提升1.1倍至1.3倍。虽然增益幅度小于LLaMA-13B,但在不同负载水平下保持一致,可归因于更低的KV Cache获取延迟和更高效的显存分配模式。延迟改善模式与LLaMA-13B一致,即使处理大规模上下文窗口时也能确保稳定的服务质量。

\textbf{可扩展性分析:}随着请求速率从1 RPS增至20 RPS,BanaServe在所有配置下均保持一致的性能优势。轻载(1--5 RPS)时性能差距主要源于BanaServe高效的调度和降低的系统开销;高负载(10--20 RPS)时,BanaServe优越的批处理策略和优化的显存管理使其维持更高吞吐量,而其他系统因资源争用和缓存利用率低下而出现性能下降。

\subsection{Azure生产负载追踪实验}

为补充合成基准测试,本实验使用Azure公开发布的LLM推理生产负载追踪数据\cite{azure2024llm}对BanaServe进行评估。该追踪数据包含匿名化的请求级日志,涵盖时间戳、输入/输出token长度和多模态请求类型,工作负载呈现出显著的异构性和突发性,反映了生产环境中真实的LLM服务条件。

图~\ref{fig:bana_azure_trace}展示了Azure生产负载追踪数据前1小时的统计特性:输入和输出长度跨越广泛范围,RPS分布呈现明显的突发性,与生产环境中非平稳到达模式一致。

\begin{figure}[htbp]
    \centering
    \subcaptionbox{输入长度分布\label{fig:bana_azure_input}}
        {\includegraphics[width=0.32\textwidth]{figures/Azure_input.pdf}}
    \hfill
    \subcaptionbox{输出长度分布\label{fig:bana_azure_output}}
        {\includegraphics[width=0.32\textwidth]{figures/Azure_output.pdf}}
    \hfill
    \subcaptionbox{请求到达速率\label{fig:bana_azure_rps}}
        {\includegraphics[width=0.32\textwidth]{figures/Azure_rps.pdf}}
    \caption{Azure生产负载追踪数据前1小时的统计特性。输入和输出长度分布呈重尾特性,请求到达速率具有显著突发性,反映了真实生产环境中多样化LLM服务工作负载的特征。}
    \label{fig:bana_azure_trace}
\end{figure}

图~\ref{fig:bana_azure_results}展示了在LLaMA-13B和OPT-13B两个模型上,BanaServe与vLLM和DistServe的吞吐量和延迟对比结果。

\begin{figure}[htbp]
    \centering
    \subcaptionbox{吞吐量对比\label{fig:bana_azure_thr}}
        {\includegraphics[width=0.48\textwidth]{figures/azure_throughput.pdf}}
    \hfill
    \subcaptionbox{平均延迟对比\label{fig:bana_azure_lat}}
        {\includegraphics[width=0.48\textwidth]{figures/azure_latency.pdf}}
    \caption{Azure生产负载追踪实验的吞吐量和延迟综合对比。BanaServe在LLaMA-13B上取得最高吞吐量(0.288 req/s)和最低延迟(3.828 s),在OPT-13B上分别为0.315 req/s和3.658 s,在重尾分布和突发流量下均显著优于vLLM和DistServe。}
    \label{fig:bana_azure_results}
\end{figure}

\textbf{Azure生产负载实验结果分析:}在LLaMA-13B和OPT-13B两个模型上,BanaServe均持续提供更高吞吐量和更低延迟,在Azure工作负载特有的重尾输入/输出分布和突发请求到达模式下性能提升尤为显著。具体而言:

\begin{itemize}
    \item \textbf{LLaMA-13B}:BanaServe吞吐量相比vLLM提升40.4\%,相比DistServe提升35.2\%;平均延迟相比vLLM降低71.2\%,相比DistServe降低27.6\%;
    \item \textbf{OPT-13B}:性能提升更为显著,吞吐量相比vLLM提升58.3\%,相比DistServe提升47.2\%;延迟相比vLLM降低74.1\%,相比DistServe降低27.4\%。
\end{itemize}

这些改进突出了BanaServe在重尾输入/输出分布和突发流量模式下维持高效率的能力。两个模型架构上一致的增益表明,系统设计能够有效泛化至不同Transformer实现,在生产风格工作负载下均表现出色。

\subsection{综合性能对比分析}

表~\ref{tab:bana_results_summary}对BanaServe在不同评估场景下的性能提升进行了系统性汇总,综合对比了相对于vLLM和DistServe的吞吐量提升和延迟降低幅度。

\begin{table}[htbp]
    \centering
    \caption{BanaServe与基线方法在不同场景下的综合性能对比}
    \label{tab:bana_results_summary}
    \begin{tabularx}{\textwidth}{p{2.8cm}p{2.0cm}AAA}
        \toprule
        \textbf{评估场景} & \textbf{模型} &
        \textbf{vs. vLLM吞吐量} &
        \textbf{vs. DistServe吞吐量} &
        \textbf{延迟降低} \\
        \midrule
        短文本(Alpaca)
            & LLaMA-13B & 1.1$\times$--1.2$\times$ & 1.1$\times$--1.2$\times$ & 较低 \\
        短文本(Alpaca)
            & OPT-13B   & 最高3.9$\times$          & 2.8$\times$--3.9$\times$ & 3.9\%--78.4\% \\
        长文本(LongBench)
            & LLaMA-13B & 最高1.5$\times$          & 1.3$\times$--1.5$\times$ & 20.6\%--65.3\% \\
        长文本(LongBench)
            & OPT-13B   & 最高1.3$\times$          & 1.1$\times$--1.3$\times$ & 稳定改善 \\
        \midrule
        Azure生产追踪
            & LLaMA-13B & $+$40.4\%               & $+$35.2\%               & $-$71.2\%(vs. vLLM) \\
        Azure生产追踪
            & OPT-13B   & $+$58.3\%               & $+$47.2\%               & $-$74.1\%(vs. vLLM) \\
        \bottomrule
    \end{tabularx}
\end{table}

实验结果验证了BanaServe设计的三个核心判断:

\begin{enumerate}
    \item \textbf{动态资源迁移对PD分离架构至关重要}:通过层级和注意力级两种粒度的动态迁移机制,BanaServe能够实时响应工作负载的不均衡性,在Prefill计算密集阶段和Decode显存密集阶段之间自适应再分配资源,从根本上消除了静态配置下的资源利用率失衡问题;

    \item \textbf{全局KV Cache存储是解耦路由与缓存的关键}:传统前缀缓存感知路由将调度决策与KV Cache物理位置紧耦合,导致持续性的负载倾斜。全局KV Cache存储通过将所有Prefill实例的缓存统一管理,配合层级流水线重叠传输($T_{KV} \approx 0.082$ ms $\ll$ $T_{F,\text{layer}} \approx 4.22$ ms),使路由器能够完全基于实时负载做出调度决策,彻底消除了缓存感知路由引发的热点问题;

    \item \textbf{粗细粒度迁移的协同设计兼顾效果与开销}:层级迁移适合处理严重的工作负载不均衡(迁移大量计算和显存资源),注意力级迁移则提供轻量级的细粒度调整(无需传输模型权重),两种机制的协同使用使BanaServe能够根据实际负载差距($\Delta$)和迁移效率比($\rho$)自适应选择最优策略。
\end{enumerate}

\section{本章小结}
\label{sec:bana_summary}

本章提出了BanaServe,一个面向PD分离架构的动态资源协同优化框架。针对现有PD分离系统中静态资源配置、内生资源不均衡和前缀缓存感知路由引发负载倾斜三类核心局限,BanaServe通过三项关键创新实现了资源分配与状态管理的解耦:

\textbf{层级权重迁移}将Transformer层的权重和KV Cache作为整体在GPU间动态迁移,实现粗粒度的计算-显存资源再平衡。迁移延迟由权重传输主导($T_{\text{layer}} \approx S^{\text{total}}_\ell / B_{\text{net}} + T_{\text{sync}}$),在高带宽数据中心互联下保持实用性,适合处理严重的工作负载不均衡场景。

\textbf{注意力级KV Cache迁移}沿注意力头维度分割KV Cache,仅传输KV状态(无模型权重),迁移延迟极低($T_{\text{attn}} \approx S_{kv}/B_{\text{net}} \ll T_{\text{layer}}$),支持在不引发显著服务中断的前提下进行实时细粒度负载均衡,与层级迁移协同构成粗细粒度自适应迁移体系。

\textbf{全局KV Cache存储}构建跨所有Prefill实例的统一KV Cache共享层,通过三阶段层级流水线(预取-计算-存储并行)隐藏通信延迟,实现通信与计算的有效重叠($T_{KV} \ll T_{F,\text{layer}}$)。路由器因此摆脱缓存位置约束,完全基于实时负载进行纯负载感知调度,从根本上消除前缀缓存感知路由引发的热点问题。

\textbf{自适应模块迁移算法}周期性监控归一化综合利用率($U_d = C_d/C^{\max}_d + M_d/M^{\max}_d$),根据过载-欠载阈值$\delta$和收益-开销比$\rho$自适应选择迁移策略,时间复杂度为$O(|D| + N_m)$,支持实时运行。

\textbf{负载感知请求调度算法}基于实时负载和队列长度对Prefill实例进行排序和调度,时间复杂度为$O(|P| \log |P| + |Q|)$,彻底摆脱了缓存局部性对调度决策的约束。

实验结果表明,相比vLLM,BanaServe在合成基准(Alpaca和LongBench)上吞吐量提升1.2倍至3.9倍,总处理时间降低3.9\%至78.4\%;相比DistServe,吞吐量提升1.1倍至2.8倍,延迟降低1.4\%至70.1\%。在Azure生产负载追踪实验中,BanaServe在重尾分布和突发流量下的鲁棒性得到进一步验证,LLaMA-13B和OPT-13B均实现了最高吞吐量和最低延迟。

然而,BanaServe目前仍存在若干局限:调度和迁移策略对异构硬件(混合精度加速器和专用通信互联)的优化支持不足;工作负载管理以被动响应为主,缺乏基于历史模式的预测性调度;系统设计局限于单区域集群,尚未支持多区域分布式部署场景。这些问题将作为未来工作的重要方向加以探索,包括硬件感知调度、基于强化学习的预测性编排,以及面向广域网优化的跨地域KV Cache同步机制。



