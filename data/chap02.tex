% !TeX root = ../sustechthesis-example.tex

% 现有LLM的推理任务批式任务调度和KV 缓存优化的方法

% 1. 现有的LLM的推理任务的批式任务调度方法
%    1. S3
%    2. continues batching
%    3. Orca 中的In-flight Batching批处理机制
%    4. 传统组合(FIFO,短作业优先,长作业优先)
% 2. 现有的KV 缓存优化方法
%    1. mooncake
%    2. memserve
%    3. vLLM
%    4. 量化
% 3. 现有方法的缺陷

\chapter{现有LLM的推理任务批式任务调度和KV缓存优化的方法}
\label{chap:related-work}

大语言模型推理优化是一个快速发展的研究领域,学术界和工业界提出了众多技术方案。本章系统梳理现有LLM推理任务的批式调度方法与KV缓存优化技术,分析各类方案的设计思想、核心机制与适用场景,并总结现有方法的局限性,为后续章节提出UELLM和BanaServe两种优化方案提供技术背景与改进动机。

\section{现有的LLM推理任务批式调度方法}
\label{sec:batch-scheduling}

批处理(Batching)是提升LLM推理吞吐量的核心手段。通过将多个请求合并处理,可以充分利用GPU的并行计算能力,提高硬件利用率。本节介绍从传统调度策略到面向LLM特性的先进批处理技术的演进过程。

\subsection{S$^3$:输出长度感知调度}
\label{subsec:s3}

S$^3$(SLO-aware, Size-based, Scheduling System)\cite{jin2023s3}是针对生成式LLM推理服务提出的批处理优化框架,其核心洞察是:\textbf{输出序列长度的差异}是导致批处理效率低下的关键因素。

传统批处理策略将不同输出长度的请求强制填充(Padding)到同一批次,导致大量无效计算。如图~\ref{fig:s3-illustration}所示,当短输出请求(如8 tokens)与长输出请求(如128 tokens)被批处理时,短请求必须等待长请求完成,造成严重的计算资源浪费。

\begin{figure}[htbp]
    \centering
    \begin{tikzpicture}[
        box/.style={draw, minimum width=2.5cm, minimum height=0.8cm, align=center, font=\small},
        short/.style={fill=green!20},
        long/.style={fill=red!20},
        arrow/.style={->, thick}
    ]
        % 传统批处理
        \node[box, short] (req1) at (0,2) {请求1\\8 tokens};
        \node[box, long] (req2) at (3,2) {请求2\\128 tokens};
        \node[box, long, minimum width=6cm] (batch1) at (1.5,0.5) {批处理结果:填充至128 tokens};
        
        \draw[arrow] (req1) -- (batch1);
        \draw[arrow] (req2) -- (batch1);
        
        \node at (1.5, -0.8) {\textbf{传统方法:}大量填充,资源浪费};
        
        % S3批处理
        \node[box, short] (req3) at (8,2) {请求1\\8 tokens};
        \node[box, long] (req4) at (11,2) {请求2\\128 tokens};
        \node[box, short] (batch2) at (8,0.5) {批次A:8 tokens};
        \node[box, long] (batch3) at (11,0.5) {批次B:128 tokens};
        
        \draw[arrow] (req3) -- (batch2);
        \draw[arrow] (req4) -- (batch3);
        
        \node at (9.5, -0.8) {\textbf{S3方法:}按长度分组,减少填充};
    \end{tikzpicture}
    \caption{S3与传统批处理策略对比}
    \label{fig:s3-illustration}
\end{figure}

S3的核心创新包括:

\textbf{(1) 输出长度预测模型}。S3采用轻量级预测器(基于输入文本特征)估计每个请求的输出行长度,将长度相近的请求归入同一批次。预测模型在多种工作负载上达到85\%以上的准确率。

\textbf{(2) 多维装箱调度算法}。将请求调度建模为多维装箱问题,综合考虑输出长度、SLO约束和GPU显存容量,采用启发式算法优化批次组合。目标函数为:
\begin{equation}
    \min \sum_{i=1}^{N} \left( \text{Padding}_i + \lambda \cdot \text{SLO\_Violation}_i \right)
\end{equation}
其中$\text{Padding}_i$为第$i$个批次的填充开销,$\text{SLO\_Violation}_i$为SLO违约惩罚,$\lambda$为权衡系数。

\textbf{(3) 动态批次重构}。在请求到达和完成时动态调整批次组成,最大化吞吐量同时保证延迟约束。

S3的局限性在于:仅优化批处理组合,未考虑模型部署配置对性能的影响;缺乏对网络拓扑异构性的感知;预测模型针对特定数据集训练,泛化能力有限。

\subsection{Continuous Batching:动态请求合并}
\label{subsec:continuous-batching}

Continuous Batching(连续批处理)\cite{kwon2023efficient}是vLLM提出的动态批处理机制,打破了传统静态批处理"等待-处理-等待"的串行模式。

\textbf{核心机制}:在Decode阶段,每当有请求完成生成并释放资源时,系统立即从等待队列中选取新请求加入当前批次,保持GPU始终处于满负荷状态。如图~\ref{fig:continuous-batching}所示,与传统静态批处理相比,Continuous Batching显著减少了GPU空闲时间。

\begin{figure}[htbp]
    \centering
    \includegraphics[width=0.9\textwidth]{example-image-a}
    \caption{Static Batching与Continuous Batching的时间线对比}
    \label{fig:continuous-batching}
\end{figure}

Continuous Batching的技术要点:

\textbf{(1) 迭代级调度}。以迭代(iteration)而非请求为粒度进行调度,每个Decode步骤完成后重新评估批次组成。

\textbf{(2) 显存动态管理}。配合PagedAttention技术(见第~\ref{subsec:vllm}节),动态分配和释放KV Cache显存,支持变长序列的高效批处理。

\textbf{(3) 抢占与恢复}。当高优先级请求到达时,可抢占低优先级请求的显存资源,被抢占请求稍后恢复执行。

Continuous Batching的优势在于高吞吐量和低延迟,但其调度决策仅基于当前系统状态,缺乏对未来负载的预测能力;同时,未考虑Prefill与Decode阶段的资源需求差异,在混合负载场景下可能出现资源竞争。

\subsection{Orca中的In-flight Batching批处理机制}
\label{subsec:orca}

Orca\cite{yu2022orca}是NVIDIA提出的LLM推理服务系统,其\textbf{In-flight Batching}(飞行中批处理)机制进一步优化了Continuous Batching,实现了Prefill与Decode阶段的流水线并行。

\textbf{关键观察}:在LLM推理中,Prefill阶段(计算密集)和Decode阶段(内存密集)对GPU资源的需求互补。传统系统串行执行两个阶段,导致资源利用率低下。

In-flight Batching的核心设计:

\textbf{(1) 阶段内批处理}。将Prefill和Decode分别批处理,但允许两个阶段在同一GPU上交替执行。当一批请求的Prefill完成后,立即开始Decode,同时GPU Tensor Core可处理下一批Prefill请求。

\textbf{(2) 细粒度同步}。采用迭代级同步而非请求级同步,减少流水线气泡(bubble)。如图~\ref{fig:inflight-batching}所示,通过精细的流水线调度,GPU计算单元和内存带宽得到更充分利用。

\begin{figure}[htbp]
    \centering
    \begin{tikzpicture}[
        timeslot/.style={draw, minimum width=1.2cm, minimum height=0.6cm, font=\tiny},
        prefill/.style={fill=red!30},
        decode/.style={fill=blue!30},
        label/.style={font=\small\bfseries}
    ]
        % 时间轴
        \foreach \x in {0,1,2,3,4,5,6,7,8} {
            \draw (\x,0) -- (\x,-0.1);
            \node[font=\tiny] at (\x,-0.3) {\x};
        }
        
        % 传统方法
        \node[label] at (-1,1.5) {传统};
        \node[timeslot, prefill] at (0.5,1.5) {P};
        \node[timeslot, prefill] at (1.5,1.5) {P};
        \node[timeslot, decode] at (2.5,1.5) {D};
        \node[timeslot, decode] at (3.5,1.5) {D};
        \node[timeslot, prefill] at (4.5,1.5) {P};
        \node[timeslot, prefill] at (5.5,1.5) {P};
        \node[timeslot, decode] at (6.5,1.5) {D};
        \node[timeslot, decode] at (7.5,1.5) {D};
        
        % In-flight Batching
        \node[label] at (-1,0.5) {Orca};
        \node[timeslot, prefill] at (0.5,0.5) {P1};
        \node[timeslot, prefill] at (1.5,0.5) {P2};
        \node[timeslot, decode, fill=blue!20] at (2,0.5) {D1};
        \node[timeslot, prefill] at (2.5,0.5) {P3};
        \node[timeslot, decode, fill=blue!40] at (3,0.5) {D1};
        \node[timeslot, decode, fill=blue!20] at (3.5,0.5) {D2};
        \node[timeslot, prefill] at (4,0.5) {P4};
        \node[timeslot, decode, fill=blue!40] at (4.5,0.5) {D2};
        \node[timeslot, decode, fill=blue!20] at (5,0.5) {D3};
        
        \draw[thick,->] (0,0) -- (8.5,0) node[right, font=\small] {时间};
    \end{tikzpicture}
    \caption{In-flight Batching流水线调度示意(P=Prefill, D=Decode)}
    \label{fig:inflight-batching}
\end{figure}

\textbf{(3) 选择性批处理}。对于Prefill阶段,仅对长度相近的请求批处理;对于Decode阶段,所有请求统一批处理,因为此时计算量与序列长度无关。

In-flight Batching的局限在于:虽然实现了阶段内并行,但未完全解耦Prefill和Decode的资源分配;缺乏对KV Cache全局共享的支持,在多实例场景下缓存命中率受限。

\subsection{传统调度策略}
\label{subsec:traditional-scheduling}

在LLM推理服务中,传统调度策略仍被广泛采用作为基准方法:

\textbf{(1) FIFO(First-In-First-Out,先进先出)}。按请求到达顺序处理,实现简单、公平性好。但FIFO对请求特性不敏感,短请求可能被长请求阻塞,导致平均等待时间较长。

\textbf{(2) SJF(Shortest Job First,短作业优先)}。优先处理预计执行时间短的请求。在LLM推理中,"短作业"对应输入长度短或输出长度短的请求。SJF可降低平均延迟,但可能导致长请求饥饿(starvation)。

\textbf{(3) LJF(Longest Job First,长作业优先)}。优先处理长请求,适用于需要尽快完成大任务的场景,但会显著增加短请求的等待时间。

\textbf{(4) 优先级调度}。根据用户等级、请求类型等分配优先级,高优先级请求优先处理。需配合抢占机制避免低优先级请求长期等待。

传统策略的共性局限在于:\textbf{静态决策},不考虑系统当前负载和资源状态;\textbf{缺乏预测},无法利用输出长度等先验信息优化调度;\textbf{粗粒度},以请求为调度单位,未探索更细粒度的资源分配。

\section{现有的KV缓存优化方法}
\label{sec:kv-cache-optimization}

KV Cache(键值缓存)是LLM推理中的核心数据结构,用于存储注意力机制中的Key和Value张量,避免重复计算。然而,KV Cache的显存占用随批次大小和序列长度线性增长,成为推理扩展的主要瓶颈。本节介绍四种代表性的KV缓存优化技术。

\subsection{Mooncake:全局KV缓存池}
\label{subsec:mooncake}

Mooncake\cite{qin2024mooncake}是月之暗面(Moonshot AI)提出的KV Cache-centric推理架构,其核心思想是将KV Cache从GPU显存解耦,构建全局共享的缓存池。

\textbf{系统架构}:Mooncake将推理集群划分为Prefill集群和Decode集群,两者之间通过高速RDMA网络连接。Prefill集群计算并产生KV Cache,Decode集群消费KV Cache进行自回归生成。关键创新在于引入\textbf{分布式KV Cache池},作为独立存储层管理所有缓存数据。

\textbf{核心技术}:

\textbf{(1) 分层缓存存储}。KV Cache池采用分层设计:热数据驻留GPU显存,温数据存放CPU内存,冷数据持久化到SSD。通过LRU(Least Recently Used)策略动态迁移数据,平衡访问延迟和存储成本。

\textbf{(2) 前缀复用}。利用请求间的前缀共享(如系统提示、多轮对话历史),通过前缀树(Trie)索引快速匹配缓存。匹配成功后,仅需计算增量部分的KV Cache,显著减少Prefill计算量。

\textbf{(3) 异步预取}。基于请求特征预测未来访问模式,异步将所需KV Cache从远程节点或下层存储预取到本地GPU,隐藏传输延迟。

Mooncake的优势在于高缓存命中率和弹性扩展能力,但其调度决策与缓存位置紧耦合:路由器必须考虑KV Cache的物理位置进行请求分发,导致负载不均衡——热点缓存节点可能过载,而冷节点资源闲置。

\subsection{MemServe:弹性内存管理}
\label{subsec:memserve}

MemServe\cite{hu2024memserve}聚焦于PD(Prefill-Decode)分离架构下的内存管理优化,提出\textbf{弹性内存池}(Elastic Memory Pool)概念。

\textbf{核心机制}:

\textbf{(1) 内存解耦}。将KV Cache存储与计算实例分离,构建独立的内存服务器集群。计算实例(GPU)通过高速网络访问远程内存,按需获取KV Cache。

\textbf{(2) 动态扩缩容}。根据负载变化动态调整内存池容量:高峰时段增加内存节点,低谷时段释放资源。结合Kubernetes等容器编排平台实现自动化运维。

\textbf{(3) 冗余消除}。识别并合并重复的KV Cache块(如共享前缀),采用引用计数管理生命周期,减少存储冗余。

MemServe的局限性在于:远程内存访问引入额外延迟,对网络带宽要求高;内存池的管理开销随规模增长,超大规模部署时可能成为瓶颈;未充分考虑Prefill与Decode实例间的负载均衡。

\subsection{vLLM:PagedAttention显存优化}
\label{subsec:vllm}

vLLM\cite{kwon2023efficient}是伯克利大学提出的开源LLM推理引擎,其\textbf{PagedAttention}技术从根本上解决了KV Cache的显存碎片问题。

\textbf{问题背景}:传统LLM推理系统将KV Cache存储为连续的显存块,如图~\ref{fig:memory-fragmentation}(a)所示。由于请求长度动态变化,频繁分配和释放导致严重的显存碎片,实际可用显存远低于物理容量。

\begin{figure}[htbp]
    \centering
    \begin{subfigure}[b]{0.45\textwidth}
        \centering
        \begin{tikzpicture}[
            block/.style={draw, minimum width=0.8cm, minimum height=0.6cm, font=\tiny}
        ]
            \foreach \x/\c in {0/white, 1/red!30, 2/red!30, 3/white, 4/white, 5/blue!30, 6/white, 7/green!30, 8/green!30, 9/white} {
                \node[block, fill=\c] at (\x*0.9, 0) {};
            }
            \node[font=\small] at (4, -1) {碎片:3块,无法合并};
        \end{tikzpicture}
        \caption{传统连续存储}
        \label{subfig:contiguous}
    \end{subfigure}
    \hfill
    \begin{subfigure}[b]{0.45\textwidth}
        \centering
        \begin{tikzpicture}[
            block/.style={draw, minimum width=0.8cm, minimum height=0.6cm, font=\tiny}
        ]
            \foreach \x/\c in {0/red!30, 1/blue!30, 2/green!30, 3/red!30, 4/green!30, 5/white, 6/white, 7/white, 8/white, 9/white} {
                \node[block, fill=\c] at (\x*0.9, 0) {};
            }
            \draw[->, thick] (5*0.9, 0.5) -- (6*0.9, 0.5);
            \node[font=\small] at (4, -1) {非连续存储,无碎片};
        \end{tikzpicture}
        \caption{PagedAttention分页存储}
        \label{subfig:paged}
    \end{subfigure}
    \caption{显存碎片问题对比}
    \label{fig:memory-fragmentation}
\end{figure}

\textbf{PagedAttention核心设计}:

\textbf{(1) 块表映射}。将KV Cache划分为固定大小的块(Block,通常为512 tokens),通过块表(Block Table)记录逻辑块到物理块的映射关系。逻辑上连续的KV Cache在物理上可以分散存储。

\textbf{(2) 动态分配}。按需分配物理块,请求完成时立即回收。由于块大小固定,回收的块可立即被其他请求复用,消除外部碎片。

\textbf{(3) 共享机制}。通过块表引用计数实现KV Cache共享:多个请求共享同一前缀时,物理块只存储一份,块表记录多个引用。Copy-on-Write机制确保写操作的安全性。

PagedAttention使vLLM的显存利用率从传统系统的40-50\%提升至90\%以上,支持更大的批次和更长的上下文。但其设计针对单实例优化,在多实例分布式场景下缺乏全局缓存共享能力。

\subsection{量化:模型压缩与加速}
\label{subsec:quantization}

量化(Quantization)通过降低模型权重和激活值的数值精度,减少显存占用和计算量,是LLM推理优化的基础技术。

\textbf{(1) 权重量化(Weight Quantization)}。将FP16/BF16精度的模型权重压缩至INT8、INT4甚至更低比特。代表性方法包括:
\begin{itemize}
    \item \textbf{LLM.int8()} \cite{dettmers2022llm}:对异常值(outliers)保持FP16精度,其余权重用INT8表示,减少精度损失。
    \item \textbf{GPTQ} \cite{frantar2022gptq}:基于近似二阶信息的逐层量化,支持INT4精度且精度损失可控。
    \item \textbf{AWQ} \cite{lin2023awq}:激活感知的权重量化,保护对激活值敏感的重要权重。
\end{itemize}

\textbf{(2) KV Cache量化}。对KV Cache进行低比特量化,显著降低长上下文场景的显存压力。例如,将KV Cache从FP16量化为INT8,显存占用减半,支持2倍长的上下文。

\textbf{(3) 混合精度量化}。根据层的重要性或激活分布动态选择精度,关键层用高精度,次要层用低精度,平衡效率和精度。

量化的优势在于通用性强,可与前述批处理、缓存优化技术正交叠加。但其局限在于:量化本身不解决资源调度和负载均衡问题;极低比特量化(如INT4以下)可能显著影响模型能力;需要硬件支持低比特运算才能发挥加速效果。

\section{现有方法的缺陷}
\label{sec:limitations}

综合上述分析,当前LLM推理优化研究在以下维度存在显著局限,这些局限性构成了本文提出UELLM和BanaServe的改进动机。

\subsection{调度粒度与资源状态的紧耦合}

现有系统(如SGLang、Mooncake)的调度决策严重依赖KV Cache的物理位置。前缀缓存感知路由(Cache-aware Routing)虽然提高了缓存命中率,但引入了\textbf{负载热点倾斜}问题:路由器被迫在计算负载均衡与缓存命中率之间做困难权衡(Cache-Load Balancing Trade-off)。

具体表现为:高缓存命中率的节点持续接收新请求,计算负载迅速饱和;而低命中率节点虽有剩余算力,却因缺乏缓存数据无法分担负载。这种\textbf{正反馈效应}导致集群资源利用率严重失衡,部分节点过载而其他节点空闲。

\subsection{资源配置的静态化与刚性约束}

现有PD分离系统(DistServe、Splitwise、MemServe)在部署时固定Prefill与Decode实例比例,无法在运行期间根据实际负载动态调整。静态配置的问题在于:

\textbf{(1) 负载失配}。实际生产环境中,Prefill和Decode的负载比例随时间动态变化(如白天交互式应用Decode密集,夜间批处理任务Prefill密集)。固定比例导致某一阶段资源过剩而另一阶段成为瓶颈。

\textbf{(2) 扩容滞后}。突发流量(Bursty Traffic)下,静态系统无法快速重新分配资源,导致队列堆积和SLO违约。

\textbf{(3) 配置搜索开销}。Morphling等配置优化方法需要对每个候选配置进行压力测试,产生巨大时间和计算开销,难以适应快速变化的负载模式。

\subsection{缺乏跨阶段的细粒度资源协同机制}

Prefill与Decode阶段的资源需求(计算vs内存)在时域和空域上互补,但现有系统缺乏在两个阶段之间实时优化资源配置的\textbf{细粒度机制}。

如图~\ref{fig:resource-imbalance}所示,Prefill实例通常计算利用率>95\%但显存利用率<40\%,Decode实例则相反(计算<40\%,显存>90\%)。这种\textbf{资源利用率失衡}导致严重的资源浪费:Prefill实例的大量显存闲置无法被Decode利用,而Decode实例的计算能力未被充分利用。

\begin{figure}[htbp]
    \centering
    \begin{tikzpicture}[
        box/.style={draw, minimum width=3cm, minimum height=2cm, align=center},
        bar/.style={draw, minimum width=0.6cm}
    ]
        % Prefill实例
        \node[box] (prefill) at (0,0) {\textbf{Prefill实例}\\[0.3cm]
            \begin{tikzpicture}[baseline]
                \node[bar, fill=red!70, minimum height=1.2cm] at (0,0) {};
                \node[bar, fill=blue!30, minimum height=0.4cm] at (0.8,0) {};
                \node[font=\tiny] at (0,-0.8) {计算95\%};
                \node[font=\tiny] at (0.8,-0.6) {显存35\%};
            \end{tikzpicture}
        };
        
        % Decode实例
        \node[box] (decode) at (5,0) {\textbf{Decode实例}\\[0.3cm]
            \begin{tikzpicture}[baseline]
                \node[bar, fill=red!30, minimum height=0.4cm] at (0,0) {};
                \node[bar, fill=blue!80, minimum height=1.1cm] at (0.8,0) {};
                \node[font=\tiny] at (0,-0.6) {计算40\%};
                \node[font=\tiny] at (0.8,-0.75) {显存95\%};
            \end{tikzpicture}
        };
        
        % 箭头表示资源浪费
        \draw[<->, thick, dashed, red] (1.5,0.5) -- (3.5,0.5) node[midway, above, font=\small] {资源互补但无法流动};
    \end{tikzpicture}
    \caption{Prefill与Decode实例资源利用率失衡}
    \label{fig:resource-imbalance}
\end{figure}

现有系统缺乏在实例间动态迁移计算任务或内存状态的能力,无法根据实时负载重新平衡资源。

\subsection{对异构性和动态性的适应性不足}

现有方案多假设同构的GPU集群和稳态负载,对以下实际场景缺乏有效支持:

\textbf{(1) 硬件异构性}。实际数据中心常混合部署不同型号GPU(如A100、H100、RTX 3090),其计算能力、显存容量、互联带宽差异显著。现有方法未充分考虑这些差异,导致资源碎片化严重。

\textbf{(2) 网络拓扑异构性}。GPU间通过NVLink、PCIe、InfiniBand等多种互联方式连接,带宽差异可达10倍以上。现有部署策略忽视拓扑结构,可能将频繁通信的层分配到高延迟链路上。

\textbf{(3) 负载动态性}。生产环境负载呈现重尾分布(Heavy-tailed)和突发特性,现有静态或准静态策略无法及时响应,导致服务质量波动。

针对上述局限,本文后续章节提出两种优化方案:第三章的\textbf{UELLM}聚焦于静态场景下的批处理优化与异构部署,通过输出长度预测和动态规划实现高效资源分配;第四章的\textbf{BanaServe}针对动态PD分离架构,通过层级资源迁移和全局KV Cache共享实现细粒度负载均衡。