% !TeX root = ../sustechthesis-example.tex

\begin{acknowledgements}

时光荏苒,转眼间三年的硕士生涯已接近尾声。2023年9月,我怀揣着对大模型的懵懂好奇踏入南科大校园;2026年6月,我带着一肚子关于KV Cache和批调度的“奇怪知识”准备离开。这三年,像是一次漫长的推理过程,有Prefill阶段的快速积累,也有Decode阶段的反复迭代,中间还夹杂着几次OOM崩溃和无数次的checkpoint回滚。所幸,最终生成了一个还算满意的输出。

首先感谢我的导师徐敏贤副研究员。徐老师兼具开放包容的学术视野与务实高效的治学风格,让我在探索大模型推理优化这个充满不确定性的领域时,既拥有自由试错的空间,又始终保持着正确的方向感。那些深夜改论文的时光,徐老师逐字逐句的批注与悉心指导,不仅打磨了我的学术表达能力,更让我深刻体会到什么是严谨的科研态度。师恩如山,铭记于心。

感谢实验室的师兄们:胡侃师兄和温林峰师兄在我初入实验室时给予了无微不至的关怀,帮助我快速完成了从本科生到研究生的“上下文切换”,顺利适应了科研节奏。感谢宋盛叶和武帅鹏,我们一同上课、一同赶DDL、一同在期末周挣扎,这些“并肩作战”的经历构成了我研究生生活最鲜活的底色。

特别感谢我的室友兼战友吴静峰。我们不仅在CoCoScale项目中并肩奋斗,更在无数个深夜讨论论文、调试代码、甚至讨论人生。那些关于学术、关于未来、关于选择的深夜长谈,让我在迷茫时找到方向,在焦虑时获得平静。那些漫步在陌生街道上的时光,是高压科研生活中最珍贵的“推理间隙”,让我得以清空缓存、重新加载。


感谢郑婉仪师妹,作为最早加入我科研项目的伙伴,我们一起完成了UELLM项目,从最初的想法到最终的实现,这段经历让我体会到协作的力量。感谢廖俊涵师弟,我们在CCF服务计算比赛中并肩作战,更在PD分离架构的探讨中碰撞出许多思想的火花,让我收获颇丰。

特别感谢胡建民师弟,与我一同在阿里爱橙科技实习。那段日子,我既要推进论文,又要面对秋招的重重压力,面试的屡次受挫让我一度陷入低谷。感谢建民的帮助让原本孤独的实习生活变得温暖而有力量。

非常感谢阿里爱橙科技AIOS团队的每一位同事。你们不仅为我提供了宝贵的工业界资源和真实的业务场景,更让我见识到了顶尖工程师的专业素养与工匠精神。这段实习经历让我明白,好的研究不仅要发论文,更要能落地。毕竟,不能部署的优化方案,就像没有显存的GPU,看着再漂亮也没用。

感谢我的父母。你们不懂什么是KV Cache、什么是投机解码,但你们懂得在我每一个重要时刻给予最坚实的支撑。我一度怀疑自己的时候,是你们无条件的信任让我有勇气再次出发。你们的爱就像最稳定的基座模型,无论我在上面做什么样的微调、遇到什么样的分布偏移,都能提供始终如一的温暖与支持。这三年,我在深圳追逐梦想,你们在远方默默守候,电话那头的一句“家里都好,你放心”,是我所有工作的最终优化目标。谢谢你们,让我在最艰难的时刻也能坚持下去,让我知道无论输出什么样的结果,总有人在等着我回家。

行文至此,窗外的荔枝花又开了。研究生的三年,是我人生中最重要的一次“预训练”。这三年教会我:真正的优化不在于消除所有延迟,而在于即使面对长尾分布的挫折,依然保持稳定的吞吐量。感谢所有给过我帮助的老师和朋友,是你们构成了我这三年最温暖的上下文。愿我们都能在各自的领域,实现更高的吞吐量与更低的延迟。

\begin{flushright}
何忆源 \\
2026年2月于安徽
\end{flushright}

\end{acknowledgements}
